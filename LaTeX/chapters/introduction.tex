\section*{Einleitung}
Dieses Skript richtet sich an gymnasiale Mittelschüler die mit den grundlegenden Begriffen Vektorgeometrie im dreidimensionalen euklidischen Raum vertraut sind.
Ziel ist es, die Vektorgeometrie zu veranschaulichen indem wir eine konkrete Anwendung in der Computergrafik betrachten, das Raytracing.
Dabei handelt es sich um eine Technik zur Generierung von realistischen 3D Bildern, welche im nächsten Kapitel genauer erklärt wird.
Wir werden den Raytracing-Algorithmus selber implementieren und damit solche Bilder generieren.
Wir verwenden dazu die Programmiersprache Python, welche in Kapitel~\ref{sec:einfuerung} kurz eingeführt wird.
Grundkenntnisse im Programmieren werden dabei vorausgesetzt.
Wer noch nie programmiert hat, kann das zum Beispiel mit der vielen online-Tutorials nachholen, bestenfalls gleich in Python.